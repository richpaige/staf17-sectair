\section{Evaluation and Conclusions}
\label{sec:conclusions}

In this paper, we provided an outline of important scalability challenges in the context of MDE, and MONDO's technical contributions for addressing them. MONDO has already contributed novel techniques and several prototype implementations in all four identified key-areas. Currently, the research contributions of MONDO are under assessment in the context of four industrial case studies. 

The first case study (provided by UNINOVA\footnote{\url{http://www.uninova.pt/}}) comes from the construction domain and involves collaborative development and automated management of large computer-aided design (CAD) models of buildings. The second case study (provided by Soft-Maint\footnote{\url{http://www.sodifrance.fr/}}) involves exploration and automated regeneration of code from large models, which have been reverse-engineered from existing legacy codebases. The third case study (provided by IKERLAN\footnote{\url{http://www.ikerlan.es/}}) involves multi-device collaborative development of models for offshore wind power generators, and the fourth case study involves managing large collections of UML models captured in a proprietary format supported by Softeam's\footnote{\url{http://www.softeam.fr/}} Modelio\footnote{\url{https://www.modeliosoft.com/}} tool. To assess the usefulness and impact of the technologies produced by MONDO, industrial partners specified a set of concrete requirements and measures with reference to existing state-of-the-art technologies during the first six months of the project. We intend to present the evaluation results in a follow-up publication as soon as the evaluation reports have been produced by the respective MONDO partners.

%In this paper we have identified a number of challenges related to scalability in Model Driven Engineering, we have discussed the state of the art in the areas of scalable language development, model querying and transformation, collaborative modelling and persistence and we have proposed directions for further research in this area which we plan to explore further in the future. %We plan to further explore these directions in the context of a dedicated 30-month research project (MONDO) which has been recently selected for funding by the European Commission and which is planned to commence in the autumn of 2013.

% \subsection*{Acknowledgements}
% We would like to thank Scott Hansen (The Open Group), Alessandra Bagnato (SOFTEAM), Pedro Mal\'o (Uninova), Vincent Hanniet (Soft-Maint) and Salvator Trujillo (IKERLAN) for their help with identifying the challenges related to scalable MDE from an industrial perspective, and for their contributions to shaping the proposed research directions.