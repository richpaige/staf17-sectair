\section{Evaluation and Conclusions}
\label{sec:conclusions}

In this paper, we have given an overview of the SECT-AIR project, which aims to reduce software cost to the UK aerospace industry. The state of play is that the cost of software for aerospace is damaging the industry, and both industry and academia must collaborate in order to introduce controls, increase productivity and hence lower costs. SECT-AIR has been designed to bring together key UK capability in development of high-integrity aerospace software.

SECT-AIR has been running since June 2016 and already has delivered a number of results, including baselining surveys and experiments to establish cross-industry state of practice, an analysis of the use of model transformation across the industry, and the development of driver technology to allow SECT-AIR partners to use modern model management technology (i.e., Epsilon) with legacy modelling tools (e.g., Artisan PTC Modeller). The focus over the next six months will be on leveraging these results to support the industry partners in more efficient development of their own profiles of UML and SysML, and on automating the generation of evidence that would be used as part of an assurance case for a high-integrity system.

% \subsection*{Acknowledgements}
% We would like to thank Scott Hansen (The Open Group), Alessandra Bagnato (SOFTEAM), Pedro Mal\'o (Uninova), Vincent Hanniet (Soft-Maint) and Salvator Trujillo (IKERLAN) for their help with identifying the challenges related to scalable MDE from an industrial perspective, and for their contributions to shaping the proposed research directions.