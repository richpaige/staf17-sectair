\section{Introduction}
\label{sec:introduction}

The majority of functionality in modern aerospace and avionics systems critically depends on software. Unlike other software domains, where tradeoffs between cost and quality can be made, quality requirements for avionics software systems are non-negotiable, fixed by standards such as DO-178C. As such, cost reductions for software have to be addressed by productivity improvements. One way to increase productivity is to better automate different engineering tasks, focusing on automating those tasks that are error-prone or repetitive, allowing engineers to focus on the challenging and creative aspects. Specific challenges that could be addressed to increase productivity and automation in avionics software engineering include:
\begin{itemize}
\item exploiting advanced architectures that support open and modular systems construction, e.g., service-oriented architectures, with the intent on reducing the burden of certification;
\item optimising the development process via enhanced automated testing techniques and streamlining the handover between systems and software engineering;
\item enhancing exploitation of model-based development, automated code generation, model-to-model transformation and automated formal analysis based on standards, so as to share development infrastructure costs and enable easier exchange of engineering artefacts;
\item improving development processes for building high-integrity devices, e.g., FPGAs, system-on-chip, multicore.
\end{itemize}
At the same time, these challenges need to be addressed in a way that makes them ready to adopt by the avionics industry, taking into account the requirements for certification.

This paper discusses how the SECT-AIR project has contributed to tackling these challenges. 
Section \ref{sec:platform-architecture} provides an overview of the overall SECT-AIR project execution. Section \ref{sec:work-packages} outline the different work packages with some focus on the model-based development tasks. Section \ref{sec:conclusions} outlines the ongoing evaluation process and concludes the paper.

